\documentclass{article}
\usepackage[utf8]{inputenc}
\usepackage[spanish]{babel}
\usepackage{listings}
\usepackage{graphicx}
\graphicspath{ {images/} }
\usepackage{cite}

\begin{document}

\begin{titlepage}
    \begin{center}
        \vspace*{1cm}
            
        \Huge
        \textbf{Nociones de la memoria del computador }
            
        \vspace{0.5cm}
        \LARGE
       Informatica II
            
        \vspace{1.5cm}
            
        {Dilan Rivas Copete}
       
        
            
        \vfill
            
        \vspace{0.8cm}
            
        \Large
        Despartamento de Ingeniería Electrónica y Telecomunicaciones\\
        Universidad de Antioquia\\
        Medellín\\
        Septiembre de 2020
            
    \end{center}
\end{titlepage}

\tableofcontents
\newpage
\section{LA MEMORIA, ¿QUE ES?}label{intro}
Teniendo en cuenta las lecturas previas,  la memoria es aquel componente físico  se encarga de retener y/o memorizar (valga la redundancia) determinados datos o cierta cantidad datos, todos esto durante un periodo de tiempo en el cual estos datos son requeridos Y/o analizados. ​La memoria es uno de los  pilares del funcionamiento de un dispositivo electrónico, esta  cumple un rol muy importante para el debido funcionamiento  de un computador, porque como ya fue dicho almacena la información requerida para el procesamiento de las instrucciones dadas por el usuario durante el uso del mismo.

La memoria permite almacenar datos de forma indefinida (El orden de horas, días semanas. meses, años, etc.) y también en cortos periodos (Segundos, nanosegundos, milésimas de segundos,etc.) esto se da principalmente durante la ejecución de algún programa o cuando se da un instrucción, la memoria guarda la información de dicha instrucción para llevarla a cabo y así mismo, existen diferentes tipos de memorias según las exigencias del usuario o según lo que se desee realizar,

\section{TIPOS DE MEMORIA} \label{contenido}
Existen una gran variedad de memorias  en el cambitode la comptacion, cada una cread y destinada para una labor especifica o que solo cada una puede cumplir,entre estas t encontramos las siguientes:
\subsection{Memoria RAM.}
La memoria RAM  de un dispositivo  es aquella donde se almacena programas datos que están siendo procesados o a la espera de ello, otra cosa que cabe desatacar o señalar es que  esta es conocida como una  memoria "volátil" , esto es debido a que los datos no se guardan de manera permanente, es eso sucede , que cuando deja de existir una fuente de energía en el dispositivo la información y/o datos se pierden. Asimismo, la memoria RAM puede ser reescrita y leída constantemente.

\subsection{Memoria ROM.}
Es la memoria qye
\subsection {Memoria Cache}
\subsection{Memoria Flash}
\subsection{Memoria Virtual}
\subsection{Memoria de video o VRAM}


\section{Gestion de la memoria} 
Las secciones (\ref{intro}), (\ref{contenido}) y (\ref{imagenes}) dependen del estilo del documento.

\bibliographystyle{IEEEtran}
\bibliography{references}

\end{document}
